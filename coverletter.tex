%THIS IS A TEMPLATE FOR COVER LETTERS
%USE IT AS A BASE

\documentclass[letterpaper,11pt]{article}

\usepackage[]{geometry}
\usepackage[empty]{fullpage}
\usepackage{parskip}
\usepackage{datetime}
\usepackage{graphicx}
\usepackage[pdftex]{hyperref}
\hypersetup{
    colorlinks,%
    citecolor=black,%
    filecolor=black,%
    linkcolor=black,%
    urlcolor=black     % can put red here to visualize the links
}
\shortdate

% Adjust margins
\addtolength{\oddsidemargin}{-0.375in}
\addtolength{\evensidemargin}{0.375in}
\addtolength{\textwidth}{0.5in}
\addtolength{\topmargin}{-.375in}
\addtolength{\textheight}{0.75in}

%-----------------------------------------------------------
% Personal Info
%\newcommand{\email}{\href{mailto:pmsolfest@gmail.com}{pmsolfest@gmail.com}}
%\newcommand{\address}{{\footnotesize 20 S. 41st St, Apt. 58, Council Bluffs, IA 51501}}
%\newcommand{\phone}{(651) 325-8903}
% NG Info
\newcommand{\email}{\href{mailto:peter.solfest@ngc.com}{peter.solfest@ngc.com}}
\newcommand{\address}{{\footnotesize 1019 Bryn Mawr Dr, Papillion, NE 68046}}
\newcommand{\phone}{(651) 325-8903}

\begin{document}

\newcommand{\mywebheader}{
\begin{tabular*}{7in}{l@{\extracolsep{\fill}}r}
  \textbf{\LARGE Peter Solfest} & \email \\
    \address & \phone  %\\
  %& Clearance Level: Secret
\end{tabular*}
\\
\vspace{0.1in}}

\mywebheader

\today

To whom it may concern,

I'm a mathematician and software engineer fascinated with space science and exploration,
able to provide a combination of technical expertise and clear communications for multiple
stakeholders while driving process improvements.

% mathematical modeling/simulations/numerical analysis/physics ed
I have a M.S. in computational mathematics which has provided me a strong background in numerical algorithms
and computational methods.
This complements my B.S. in physics, providing a solid foundation in my professional career with which I have
developed algorithms for assimilating observational data into ionospheric models, created
modeling tools to simulate weather effects on radio communications, and helped lead the
development efforts for maintaining and improving the Air Force's Space Weather Analysis and Forecasting System (SWAFS).

% data analysis + visualization
I have found that one of the most interesting challenges when developing new models
or creating simulations is generating visualizations that not only demonstrate the model's effectiveness, but 
highlight only the relevant features.
While this is rarely easy, it is immensely satisfying to accomplish.
% communication skills
I also take pride in my ability to concisely and clearly create documentation following the same principles,
clearly and exactly communicating the message without containing miscellaneous tangents or
details irrelevant for the target audience that would pull focus away from the core message.


% quick study - ground up learning, broad interest, cutting edge
% troubleshooting (live data feed, ops support, tech lead supporting other devs)
When I was hired by Northrop Grumman to work on SWAFS, I began as a software
developer implementing new models and features while maintaining the existing capabilities.
This has included implementing the ingest of live data feeds an providing operational support.
But my curiosity about the models themselves led to working with our university partners to improving one of our major scientific models.
Furthermore, I have a deep drive to continuously improve everything, which led to heavy involvement
with defining and improving the agile processes and documentation used by the team.
This resulted in a promotion to the role of tech lead, where I have continued to perform development
activities, but also been heavily involved in process experimentation and improvements (both for the
SWAFS team as well as the larger SEMS enterprise), 
designing of new features, and providing technical input for contractual updates.

To better support the design and documentation of new features, I have leveraged the training opportunities
from Northrop Grumman (through CalTech) to learn systems engineering and systems modeling, while obtaining
industry certifications for secure software development.
This led to designing multiple new capabilities in SWAFS using a model-based systems engineering approach.

While working on SWAFS, I also worked in the NG Weather and Space Impacts Research and Development Center
to model weather effects on radio communications.
During this work, I set up the SharePoint and GitHub sites used by the team to improve
the collaboration opportunities across sectors and within the team.
These tools also enabled the usage of modern, flexible coding processes within the
research group.

% conclusion - complete package
Along with the technical expertise and communications skills above, I can provide a strong
sense of enthusiasm to this role. I hope to hear from you soon.


Sincerely,

\includegraphics[height=.5in]{signature.png}

Peter M. Solfest
\end{document}
