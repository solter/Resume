%THIS IS A TEMPLATE FOR COVER LETTERS
%USE IT AS A BASE

\documentclass[letterpaper,12pt]{article}

\usepackage[]{geometry}
\usepackage{parskip}
\usepackage{datetime}
\usepackage{graphicx}
\shortdate

% Adjust margins
\addtolength{\oddsidemargin}{-0.375in}
\addtolength{\evensidemargin}{0.375in}
\addtolength{\textwidth}{0.5in}
\addtolength{\topmargin}{-.375in}
\addtolength{\textheight}{0.75in}

\begin{document}

Peter M. Solfest\\
7608 Peltier Lake Drive\\
Lino Lakes, MN 55038\\
phone: (651) 325-8903\\
email: pmsolfest@gmail.com

\today

%other person's info?
%Name, Title
%Company
%Address
%City, State, zip

To whom it may concern, %TODO: change this to specific person

%Introduction
I have recently graduated with a master's degree from Michigan Tech in applied mathematics after
completing a B.S. in physics and mathematics.
I am pursuing a job with which to employ my computational mathematics knowledge
while using my background in physics.

%argument why I'm a good fit
Utilizing my entire educational background will allow me to efficiently
contribute to Lincoln laboratory.
As a graduate student in mathematics, one of my primary interests was the study of numerical methods,
which will be leveraged to contribute to the development and usage of efficient software.
My education and internship have also honed my skill at writing technical reports.
%%These communication skills have been further honed through teaching courses to undergraduates
%%throughout my education.
%If strong connection to physics
Furthermore, my undergraduate degree in physics has given me a solid understanding of the
physical processes which lie at the heart of modeling software and mathematics.

%Closing
My educational background in both mathematics and physics will allow 
me to approach positions with the technical skills required and a thorough understanding
of the underlying physical assumptions.
Thank you for your time and consideration, and hope to hear from you soon.

Sincerely,

\includegraphics[height=.5in]{signature.png}

Peter M. Solfest

\end{document}
